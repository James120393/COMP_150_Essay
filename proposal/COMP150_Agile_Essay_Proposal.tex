\documentclass{scrartcl}

\usepackage[hidelinks]{hyperref}
\usepackage[none]{hyphenat}

\title{Essay Proposal}
\subtitle{COMP110 - Computer Architecture Essay}

\author{JH182233}

\begin{document}


\maketitle

\section*{Topic}

My essay will be on

Procedural Level Generation For A 2D Platform Game.

\section*{Paper 1}
\begin{description}
\item[Title:] PCG-based game design: creating Endless Web
\item[Citation:] \cite{web}
\item[Abstract:] ``This paper describes the creation of the game Endless Web, a 2D platforming game in which the player's actions determine the ongoing creation of the world she is exploring. Endless Web is an example of a PCG-based game: it uses procedural content generation (PCG) as a mechanic, and its PCG system, Launchpad, greatly influenced the aesthetics of the game. All of the player's strategies for the game revolve around the use of procedural content generation. Many design challenges were encountered in the design and creation of Endless Web, for both the game and modifications that had to be made to Launchpad. These challenges arise largely from a loss of fine-grained control over the player's experience; instead of being able to carefully craft each element the player can interact with, the designer must instead craft algorithms to produce a range of content the player might experience. In this paper we provide a definition of PCG-based game design and describe the challenges faced in creating a PCG-based game. We offer our solutions, which impacted both the game and the underlying level generator, and identify issues which may be particularly important as this area matures.''
\item[Web link:] \url{http://dl.acm.org/citation.cfm?id=2282338.2282375&coll=DL&dl=GUIDE&CFID=728180100&CFTOKEN=26846601}
\item[Full text link:] \url{http://dl.acm.org/citation.cfm?id=2282338.2282375&coll=DL&dl=GUIDE&CFID=728180100&CFTOKEN=26846601}
\item[Comments:] This is a seminal article, which laid the foundations for what we now call information theory.
	It has been cited more than $76\,000$ times, making it one of the most-cited articles in computer science.
\end{description}

\section*{Paper 2}
\begin{description}
\item[Title:] Title of paper
\item[Citation:] \cite{bibtex_key}
\item[Abstract:] Copy and paste the abstract here
\item[Web link:] Give the URL of the paper in IEEE Xplore, ACM Digital Library, or similar
\item[Full text link:] Give the URL of a downloadable PDF of the paper, if you can find one
\item[Comments:] Write a few sentences on how you found the article and why you believe it is relevant and/or important.
\end{description}

\section*{Paper 3}
\begin{description}
\item[Title:] Title of paper
\item[Citation:] \cite{bibtex_key}
\item[Abstract:] Copy and paste the abstract here
\item[Web link:] Give the URL of the paper in IEEE Xplore, ACM Digital Library, or similar
\item[Full text link:] Give the URL of a downloadable PDF of the paper, if you can find one
\item[Comments:] Write a few sentences on how you found the article and why you believe it is relevant and/or important.
\end{description}

\section*{Paper 4}
\begin{description}
\item[Title:] Title of paper
\item[Citation:] \cite{bibtex_key}
\item[Abstract:] Copy and paste the abstract here
\item[Web link:] Give the URL of the paper in IEEE Xplore, ACM Digital Library, or similar
\item[Full text link:] Give the URL of a downloadable PDF of the paper, if you can find one
\item[Comments:] Write a few sentences on how you found the article and why you believe it is relevant and/or important.
\end{description}

\section*{Paper 5}
\begin{description}
\item[Title:] Title of paper
\item[Citation:] \cite{bibtex_key}
\item[Abstract:] Copy and paste the abstract here
\item[Web link:] Give the URL of the paper in IEEE Xplore, ACM Digital Library, or similar
\item[Full text link:] Give the URL of a downloadable PDF of the paper, if you can find one
\item[Comments:] Write a few sentences on how you found the article and why you believe it is relevant and/or important.
\end{description}

\bibliographystyle{ieeetr}
\bibliography{comp110_architecture}

\end{document}
